\documentclass{article}\usepackage[]{graphicx}\usepackage[]{color}
%% maxwidth is the original width if it is less than linewidth
%% otherwise use linewidth (to make sure the graphics do not exceed the margin)
\makeatletter
\def\maxwidth{ %
  \ifdim\Gin@nat@width>\linewidth
    \linewidth
  \else
    \Gin@nat@width
  \fi
}
\makeatother

\definecolor{fgcolor}{rgb}{0.345, 0.345, 0.345}
\newcommand{\hlnum}[1]{\textcolor[rgb]{0.686,0.059,0.569}{#1}}%
\newcommand{\hlstr}[1]{\textcolor[rgb]{0.192,0.494,0.8}{#1}}%
\newcommand{\hlcom}[1]{\textcolor[rgb]{0.678,0.584,0.686}{\textit{#1}}}%
\newcommand{\hlopt}[1]{\textcolor[rgb]{0,0,0}{#1}}%
\newcommand{\hlstd}[1]{\textcolor[rgb]{0.345,0.345,0.345}{#1}}%
\newcommand{\hlkwa}[1]{\textcolor[rgb]{0.161,0.373,0.58}{\textbf{#1}}}%
\newcommand{\hlkwb}[1]{\textcolor[rgb]{0.69,0.353,0.396}{#1}}%
\newcommand{\hlkwc}[1]{\textcolor[rgb]{0.333,0.667,0.333}{#1}}%
\newcommand{\hlkwd}[1]{\textcolor[rgb]{0.737,0.353,0.396}{\textbf{#1}}}%
\let\hlipl\hlkwb

\usepackage{framed}
\makeatletter
\newenvironment{kframe}{%
 \def\at@end@of@kframe{}%
 \ifinner\ifhmode%
  \def\at@end@of@kframe{\end{minipage}}%
  \begin{minipage}{\columnwidth}%
 \fi\fi%
 \def\FrameCommand##1{\hskip\@totalleftmargin \hskip-\fboxsep
 \colorbox{shadecolor}{##1}\hskip-\fboxsep
     % There is no \\@totalrightmargin, so:
     \hskip-\linewidth \hskip-\@totalleftmargin \hskip\columnwidth}%
 \MakeFramed {\advance\hsize-\width
   \@totalleftmargin\z@ \linewidth\hsize
   \@setminipage}}%
 {\par\unskip\endMakeFramed%
 \at@end@of@kframe}
\makeatother

\definecolor{shadecolor}{rgb}{.97, .97, .97}
\definecolor{messagecolor}{rgb}{0, 0, 0}
\definecolor{warningcolor}{rgb}{1, 0, 1}
\definecolor{errorcolor}{rgb}{1, 0, 0}
\newenvironment{knitrout}{}{} % an empty environment to be redefined in TeX

\usepackage{alltt}
\IfFileExists{upquote.sty}{\usepackage{upquote}}{}
\begin{document}
\SweaveOpts{concordance=TRUE}
 
Загрузка данных и их открытие данных 

\begin{knitrout}
\definecolor{shadecolor}{rgb}{0.969, 0.969, 0.969}\color{fgcolor}\begin{kframe}
\begin{alltt}
\hlkwd{library}\hlstd{(}\hlstr{"edf"}\hlstd{)}
\hlkwd{library}\hlstd{(}\hlstr{"tidyverse"}\hlstd{)}
\end{alltt}


{\ttfamily\noindent\itshape\color{messagecolor}{\#\# Loading tidyverse: ggplot2\\\#\# Loading tidyverse: tibble\\\#\# Loading tidyverse: tidyr\\\#\# Loading tidyverse: readr\\\#\# Loading tidyverse: purrr\\\#\# Loading tidyverse: dplyr}}

{\ttfamily\noindent\itshape\color{messagecolor}{\#\# Conflicts with tidy packages ----------------------------------------------}}

{\ttfamily\noindent\itshape\color{messagecolor}{\#\# filter(): dplyr, stats\\\#\# lag():\ \ \ \ dplyr, stats}}\begin{alltt}
\hlstd{openEeg} \hlkwb{<-} \hlkwa{function}\hlstd{(}\hlkwc{shortfilename}\hlstd{)\{}
  \hlstd{datafile} \hlkwb{=} \hlkwd{paste}\hlstd{(}\hlkwd{getwd}\hlstd{(), shortfilename,} \hlkwc{sep} \hlstd{=} \hlstr{'/'}\hlstd{)}
  \hlstd{eeg} \hlkwb{=} \hlkwd{read.edf}\hlstd{(datafile)}
  \hlkwd{return}\hlstd{(eeg)}
\hlstd{\}}

\hlstd{structEegData} \hlkwb{<-} \hlkwa{function}\hlstd{(}\hlkwc{eeg}\hlstd{)\{}
  \hlstd{len} \hlkwb{=} \hlkwd{length}\hlstd{(eeg}\hlopt{$}\hlstd{events}\hlopt{$}\hlstd{annotation)}
  \hlstd{numberOfStimuli} \hlkwb{=} \hlkwd{grep}\hlstd{(}\hlstr{"\textbackslash{}\textbackslash{}([12]\textbackslash{}\textbackslash{})"}\hlstd{, eeg}\hlopt{$}\hlstd{events}\hlopt{$}\hlstd{annotation)}
  \hlstd{stimuliCount} \hlkwb{=} \hlkwd{length}\hlstd{(numberOfStimuli)}

  \hlstd{furiePointsCount} \hlkwb{=} \hlnum{70}\hlstd{;}
  \hlstd{output} \hlkwb{=} \hlkwd{numeric}\hlstd{(stimuliCount)}
  \hlstd{input} \hlkwb{=} \hlkwd{matrix}\hlstd{(}\hlkwc{data} \hlstd{=} \hlnum{0}\hlstd{,} \hlkwc{nrow} \hlstd{= stimuliCount,} \hlkwc{ncol} \hlstd{=} \hlnum{21}\hlopt{*}\hlstd{furiePointsCount}\hlopt{*}\hlnum{2}\hlstd{)}

  \hlstd{sampleRate} \hlkwb{=} \hlstd{eeg}\hlopt{$}\hlstd{header.signal}\hlopt{$}\hlstd{EEG_Fp1}\hlopt{$}\hlstd{samplingrate;}
  \hlstd{samplesCount} \hlkwb{=} \hlkwd{round}\hlstd{(sampleRate} \hlopt{*} \hlnum{0.5}\hlstd{)}

  \hlstd{counter} \hlkwb{=} \hlnum{0}\hlstd{;}

  \hlkwa{for} \hlstd{(i} \hlkwa{in} \hlstd{numberOfStimuli)}
  \hlstd{\{}
    \hlstd{counter} \hlkwb{=} \hlstd{counter} \hlopt{+} \hlnum{1}
    \hlstd{eegSample} \hlkwb{=} \hlkwd{numeric}\hlstd{(}\hlnum{21}\hlopt{*}\hlstd{furiePointsCount}\hlopt{*}\hlnum{2}\hlstd{)}
    \hlstd{onSet} \hlkwb{=} \hlkwd{round}\hlstd{((eeg}\hlopt{$}\hlstd{events}\hlopt{$}\hlstd{onset[i]} \hlopt{-} \hlnum{0.05}\hlstd{)}\hlopt{*}\hlstd{sampleRate)}
    \hlkwa{for} \hlstd{(j} \hlkwa{in} \hlnum{1}\hlopt{:}\hlnum{21}\hlstd{)}
    \hlstd{\{}
      \hlstd{buffer} \hlkwb{=} \hlstd{eeg}\hlopt{$}\hlstd{signal[[j]]}\hlopt{$}\hlstd{data[onSet}\hlopt{:}\hlstd{(onSet} \hlopt{+} \hlstd{samplesCount} \hlopt{-} \hlnum{1}\hlstd{)]}
      \hlstd{furie} \hlkwb{=} \hlkwd{fft}\hlstd{(buffer,} \hlkwc{inverse} \hlstd{=} \hlnum{FALSE}\hlstd{)}
      \hlstd{eegSample[(}\hlnum{1} \hlopt{+} \hlstd{(j} \hlopt{-} \hlnum{1}\hlstd{)}\hlopt{*}\hlstd{furiePointsCount}\hlopt{*}\hlnum{2}\hlstd{)}\hlopt{:}\hlstd{((j}\hlopt{*}\hlnum{2} \hlopt{-} \hlnum{1}\hlstd{)}\hlopt{*}\hlstd{furiePointsCount)]} \hlkwb{<-}
        \hlkwd{log10}\hlstd{(}\hlkwd{Mod}\hlstd{(furie[}\hlnum{1}\hlopt{:}\hlstd{furiePointsCount]))}
      \hlstd{eegSample[(}\hlnum{1} \hlopt{+} \hlstd{(j}\hlopt{*}\hlnum{2} \hlopt{-} \hlnum{1}\hlstd{)}\hlopt{*}\hlstd{furiePointsCount)}\hlopt{:}\hlstd{(j}\hlopt{*}\hlstd{furiePointsCount}\hlopt{*}\hlnum{2}\hlstd{) ]} \hlkwb{<-}
        \hlkwd{Arg}\hlstd{(furie[}\hlnum{1}\hlopt{:}\hlstd{furiePointsCount])}
    \hlstd{\}}

    \hlstd{input[counter,]} \hlkwb{=} \hlstd{eegSample}
    \hlstd{output[counter]} \hlkwb{=} \hlkwd{as.numeric}\hlstd{(}\hlkwd{grepl}\hlstd{(}\hlstr{"\textbackslash{}\textbackslash{}(1\textbackslash{}\textbackslash{})"}\hlstd{, eeg}\hlopt{$}\hlstd{events}\hlopt{$}\hlstd{annotation[i]));}
  \hlstd{\}}
  \hlstd{data} \hlkwb{=} \hlkwd{list}\hlstd{(}\hlkwc{x} \hlstd{= input,} \hlkwc{y} \hlstd{= output);}
  \hlkwd{return}\hlstd{(data)}
\hlstd{\}}

\hlstd{dataFrame1} \hlkwb{<-} \hlstr{"Мухуров/170417_0014_EEG.edf"} \hlopt \hlstd{openEeg} \hlopt \hlstd{structEegData}
\hlstd{dataFrame2} \hlkwb{<-} \hlstr{"Смирнов/24042017.edf"} \hlopt \hlstd{openEeg} \hlopt \hlstd{structEegData}
\end{alltt}
\end{kframe}
\end{knitrout}


\begin{knitrout}
\definecolor{shadecolor}{rgb}{0.969, 0.969, 0.969}\color{fgcolor}\begin{kframe}
\begin{alltt}
\hlkwd{library}\hlstd{(xtable)}
\hlkwd{data}\hlstd{(}\hlstr{"tli"}\hlstd{)}
\hlstd{variable1} \hlkwb{<-} \hlnum{1}
\hlstd{variable2} \hlkwb{=} \hlnum{2}
\hlstd{hello_txt} \hlkwb{<-} \hlstr{"Hello world"} \hlcom{# just to illustrate the markup}
\end{alltt}
\end{kframe}
\end{knitrout}

I've now created two variables, one with the value 1
and one with the value 2. I've used two different 
assignment operators, the $<-$ and the $=$. The $<-$ is preferred because 
it gives a natural understanding of assignment since the $<-$ looks 
like an arrow while $=$ can be confused with equal (that usually 
is represented by two equal signs "$==$").

\begin{knitrout}
\definecolor{shadecolor}{rgb}{0.969, 0.969, 0.969}\color{fgcolor}\begin{kframe}
\begin{verbatim}
## % latex table generated in R 3.4.0 by xtable 1.8-2 package
## % Thu Apr 27 09:02:52 2017
## \begin{table}[ht]
## \centering
## \begin{tabular}{rrlllr}
##   \hline
##  & grade & sex & disadvg & ethnicty & tlimth \\ 
##   \hline
## 1 &   6 & M & YES & HISPANIC &  43 \\ 
##   2 &   7 & M & NO & BLACK &  88 \\ 
##   3 &   5 & F & YES & HISPANIC &  34 \\ 
##   4 &   3 & M & YES & HISPANIC &  65 \\ 
##   5 &   8 & M & YES & WHITE &  75 \\ 
##   6 &   5 & M & NO & BLACK &  74 \\ 
##   7 &   8 & F & YES & HISPANIC &  72 \\ 
##   8 &   4 & M & YES & BLACK &  79 \\ 
##   9 &   6 & M & NO & WHITE &  88 \\ 
##   10 &   7 & M & YES & HISPANIC &  87 \\ 
##    \hline
## \end{tabular}
## \end{table}
\end{verbatim}
\end{kframe}
\end{knitrout}


\end{document}
